%%%%%%%%%%%%%%%%%%%%%%%%%%
%% Folie: Discussion    %%
%%%%%%%%%%%%%%%%%%%%%%%%%%
\begin{frame}
    \frametitle{Discussion}
	\vspace*{0.8cm}
    \textbf{Positiv}

\begin{PraesentationAufzaehlung}
	\item Relative error $< 3\%$
	
	\item Convolutions paired with an encoder decoder structure seem to \newline catch regions of interest fast and reliable

    \item U-Nets can outperform LSTM’s in accuracy as well as in speed \newline with a fraction of capacity (in time-series problems)
    
    \item accuracy does not suffer too much from models with a lot less capacity \newline mostly affects sharpness of solutions

    \item Inference speed is $1000\times$ when compared with OpenFOAM solver
    
    \item Accuracy improvements still possible (bigger models, more training data)
\end{PraesentationAufzaehlung}

\end{frame}
\clearpage

\begin{frame}
    \frametitle{Discussion}
	\vspace*{0.8cm}
    \textbf{Negativ}

\begin{PraesentationAufzaehlung}
\item Proir knowledge needed for proper pre-processing

\item Solvers needed for dataset generation

\item Extrapolation yields mediocre results

\item Fresh Training needed for other shapes (e.g. cars in wind tunnel) \newline -- transfer learning unlikely

\item Trade off: training speed -- grid resolution

\item Possible data loss from transformation: \newline adaptive grid (solver) $\implies$ cartesian grid (NN)

\item no guarantee for correctness

\item Accuracy improvements computationally expensive likely requires \newline tailored architectures and loss functions
\end{PraesentationAufzaehlung}
\end{frame}
\clearpage