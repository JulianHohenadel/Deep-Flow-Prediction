\begin{frame}
    \frametitle{Generalization}
    \vspace*{0.8cm}
Motivation: Key question in deep learning: How well does my NN perform on unseen data?


\includegraphics[width=0.85\textwidth, height=.5\textheight]{./Ressourcen/Praesentation/Bilder/generalization.png}%

\vspace*{-0.6cm}

Taken from Deep Learning by Adam Gibson, Josh Patterson, O'Reily Media Inc., 2017

\end{frame}
\clearpage

\begin{frame}
    \frametitle{Generalization}
    \vspace*{0.8cm}
\begin{multicols}{2}
    Splitted up training data:
    \begin{PraesentationAufzaehlung}
		\item Regular
		\item Mixed ($50\%$ regular, $50\%$ sheared ($\pm 15$ degrees))
	\end{PraesentationAufzaehlung}
	
	The plot shows training with a $30.9 \cdot 10^6$ parameter model
	
	The high capacity supports training with the mixed dataset, \newline
	achieving a even lower error
	
	 
	
    \vfill\columnbreak
	\includegraphics[width=\columnwidth, height=.6\textheight]{./Ressourcen/Praesentation/Bilder/generaliz_plot.png}
\end{multicols}
\vspace*{-1.4cm}
    Taken from \url{https://arxiv.org/pdf/1810.08217.pdf}
\end{frame}
\clearpage



\begin{frame}
    \frametitle{Generalization -- Evaluation}
    \vspace*{0.8cm}

Generalization with different angels of attack: \\[\baselineskip]

\includegraphics[width=.5\textwidth, height=.4\textheight]{./Ressourcen/Praesentation/Bilder/aoa.png}%
\newline

Taken from \url{http://www.aviationchief.com/angle-of-attack.html}

\end{frame}
\clearpage

%%%%%%%%%%%%%%%%%%%%%%%%%%%%%%%%%%%%%%%%%%%%%%%%%%%%%
%% Folie: Diagramm                                 %%
%%%%%%%%%%%%%%%%%%%%%%%%%%%%%%%%%%%%%%%%%%%%%%%%%%%%%
\begin{frame}
    \frametitle{Generalization -- Evaluation}
Error increase of different angle of attack intervals as test set wrt. ground truth $[-22.5 , 22.5]$
\begin{center}
	\vspace*{-1cm}
    \begin{tikzpicture}
        \begin{axis}[
                ybar=10,
                bar width=27.1,
                axis line style={transparent},
                every tick/.style={transparent},
                enlarge x limits=0.145, % X-Achse skalieren
                clip limits=true,
                ymin=0,
                ymax=30,
                width=\textwidth,
                height=.65\textheight,
                symbolic x coords={-45 -- 45,-90 -- 90,-67.5 -- -22.5,22.5 -- 67.5},
                xticklabels={-45 -- 45,-90 -- 90,-67.5 -- -22.5,22.5 -- 67.5},
                xtick=data,
                ytick={0,5,10,15,20,25},
                every tick label/.append style={font=\fontsize{13}{14}\selectfont},
                ymajorgrids,
                legend image code/.code={\draw[draw=none] (0cm,-0.12cm) rectangle (0.29cm,0.17cm);}, % Legenden-Symbol  
                legend columns=3,
                reverse legend,
                legend style={
                    font={\usebeamerfont{footnote}},
                    fill=none,
                    draw=none,
                    /tikz/every odd column/.append style={column sep=0.07cm}, % Abstand zwischen Legenden-Symbol
                    /tikz/every even column/.append style={column sep=0.8cm} % Abstand zwischen den Legendeneinträgen
                 },
                legend to name=PraesentationDiagrammVertikalLegende
            ]
            
            \addlegendentry{Combined}        
            \addlegendentry{Pressure}    
            \addlegendentry{Velocity}    
            
            \addplot[color=TUMBlauDunkel, fill=TUMBlauDunkel] coordinates {
                (-45 -- 45,6.07)
                (-90 -- 90,18.23)
                (-67.5 -- -22.5,15.11)
                (22.5 -- 67.5,22.99)
            };
            
            \addplot[color=TUMBlauHell, fill=TUMBlauHell] coordinates {
                (-45 -- 45,3.66)
                (-90 -- 90,5.73)
                (-67.5 -- -22.5,6.00)
                (22.5 -- 67.5,10.70)
            };
            
            \addplot[color=TUMBlauMittel, fill=TUMBlauMittel] coordinates {
                (-45 -- 45,5.14)
                (-90 -- 90,19.20)
                (-67.5 -- -22.5,13.84)
                (22.5 -- 67.5,23.25)
            };        
        \end{axis}
    \end{tikzpicture}

    \vspace*{-5mm}
    \ref*{PraesentationDiagrammVertikalLegende}%
\end{center}
\end{frame}
\clearpage

\begin{frame}
    \frametitle{Generalization -- $[-22.5, 22.5]$}
    \vspace*{.1cm}
\begin{multicols}{2}
	
	\includegraphics[width=.9\columnwidth, height=.6\textheight]{./Ressourcen/Praesentation/Bilder/TransferEval/std/0065_bw.png}%
    \vfill\columnbreak
    
    \includegraphics[width=.9\columnwidth, height=.6\textheight]{./Ressourcen/Praesentation/Bilder/TransferEval/std/0084_bw.png}%
\end{multicols}
    
\end{frame}
\clearpage

\begin{frame}
    \frametitle{Generalization -- $[-45, 45]$}
    \vspace*{.1cm}
\begin{multicols}{2}
	
	\includegraphics[width=.9\columnwidth, height=.6\textheight]{./Ressourcen/Praesentation/Bilder/TransferEval/45/0088_bw.png}%
    \vfill\columnbreak
    
    \includegraphics[width=.9\columnwidth, height=.6\textheight]{./Ressourcen/Praesentation/Bilder/TransferEval/45/0089_bw.png}%
\end{multicols}
    
\end{frame}
\clearpage

\begin{frame}
    \frametitle{Generalization -- $[-90, 90]$}
    \vspace*{.1cm}
\begin{multicols}{2}
	
	\includegraphics[width=.9\columnwidth, height=.6\textheight]{./Ressourcen/Praesentation/Bilder/TransferEval/90/0035_bw.png}%
    \vfill\columnbreak
    
    \includegraphics[width=.9\columnwidth, height=.6\textheight]{./Ressourcen/Praesentation/Bilder/TransferEval/90/0067_bw.png}%
\end{multicols}
    
\end{frame}
\clearpage

\begin{frame}
    \frametitle{Generalization -- $[-67.5, -22.5]$}
    \vspace*{.1cm}
\begin{multicols}{2}
	
	\includegraphics[width=.9\columnwidth, height=.6\textheight]{./Ressourcen/Praesentation/Bilder/TransferEval/left/0064_bw.png}%
    \vfill\columnbreak
    
    \includegraphics[width=.9\columnwidth, height=.6\textheight]{./Ressourcen/Praesentation/Bilder/TransferEval/left/0074_bw.png}%
\end{multicols}
    
\end{frame}
\clearpage

\begin{frame}
    \frametitle{Generalization -- $[22.5, 67.5]$}
    \vspace*{.1cm}
\begin{multicols}{2}
	
	\includegraphics[width=.9\columnwidth, height=.6\textheight]{./Ressourcen/Praesentation/Bilder/TransferEval/right/0088_bw.png}%
    \vfill\columnbreak
    
    \includegraphics[width=.9\columnwidth, height=.6\textheight]{./Ressourcen/Praesentation/Bilder/TransferEval/right/0089_bw.png}%
\end{multicols}
    
\end{frame}
\clearpage

%%%%%%%%%%%%%%%%%%%%%%%%%%%%%%%%%%%%%%%%%%%%%%%%%%%%%
%% Folie: Diagramm                                 %%
%%%%%%%%%%%%%%%%%%%%%%%%%%%%%%%%%%%%%%%%%%%%%%%%%%%%%
\begin{frame}
    \frametitle{Generalization -- Evaluation}
Error increase of different angle of attack interpolations (test set) wrt. ground truth $[-22.5 , 22.5]$
\begin{center}
	\vspace*{-1cm}
    \begin{tikzpicture}
        \begin{axis}[
                ybar=10,
                bar width=27.1,
                axis line style={transparent},
                every tick/.style={transparent},
                enlarge x limits=0.145, % X-Achse skalieren
                clip limits=true,
                ymin=0,
                ymax=30,
                width=\textwidth,
                height=.65\textheight,
                symbolic x coords={-22.5 -- -10,10 -- 22.5},
                xticklabels={-22.5 -- -10,10 -- 22.5},
                xtick=data,
                ytick={0,5,10,15,20,25},
                every tick label/.append style={font=\fontsize{13}{14}\selectfont},
                ymajorgrids,
                legend image code/.code={\draw[draw=none] (0cm,-0.12cm) rectangle (0.29cm,0.17cm);}, % Legenden-Symbol  
                legend columns=3,
                reverse legend,
                legend style={
                    font={\usebeamerfont{footnote}},
                    fill=none,
                    draw=none,
                    /tikz/every odd column/.append style={column sep=0.07cm}, % Abstand zwischen Legenden-Symbol
                    /tikz/every even column/.append style={column sep=0.8cm} % Abstand zwischen den Legendeneinträgen
                 },
                legend to name=PraesentationDiagrammVertikalLegende
            ]
            
            \addlegendentry{Combined}        
            \addlegendentry{Pressure}    
            \addlegendentry{Velocity}    
            
            %Combined
            \addplot[color=TUMBlauDunkel, fill=TUMBlauDunkel] coordinates {
                (-22.5 -- -10,11.38)
                (10 -- 22.5,9.37)
            };
            
            %Pressure
            \addplot[color=TUMBlauHell, fill=TUMBlauHell] coordinates {
                (-22.5 -- -10,4.64)
                (10 -- 22.5,2.5)
            };
            
            %Velocity
            \addplot[color=TUMBlauMittel, fill=TUMBlauMittel] coordinates {
                (-22.5 -- -10,12.66)
                (10 -- 22.5,10.13)
            };        
        \end{axis}
    \end{tikzpicture}

    \vspace*{-5mm}
    \ref*{PraesentationDiagrammVertikalLegende}%
\end{center}
\end{frame}
\clearpage
%TODO

\begin{frame}
    \frametitle{Generalization -- training: $[-10, 22.5]$, testing: $[-22.5, -10]$}
    \vspace*{.1cm}
\begin{multicols}{2}
	
	\includegraphics[width=.9\columnwidth, height=.6\textheight]{./Ressourcen/Praesentation/Bilder/TransferEval/interpolation/left_test/0085_bw.png}%
    \vfill\columnbreak
    
    \includegraphics[width=.9\columnwidth, height=.6\textheight]{./Ressourcen/Praesentation/Bilder/TransferEval/interpolation/left_test/0086_bw.png}%
\end{multicols}
    
\end{frame}
\clearpage

\begin{frame}
    \frametitle{Generalization -- training: $[-22.5, 10]$, testing: $[10, 22.5]$}
    \vspace*{.1cm}
\begin{multicols}{2}
	
	\includegraphics[width=.9\columnwidth, height=.6\textheight]{./Ressourcen/Praesentation/Bilder/TransferEval/interpolation/right_test/0019_bw.png}%
    \vfill\columnbreak
    
    \includegraphics[width=.9\columnwidth, height=.6\textheight]{./Ressourcen/Praesentation/Bilder/TransferEval/interpolation/right_test/0058_bw.png}%
\end{multicols}
    
\end{frame}
\clearpage
